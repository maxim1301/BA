\chapter[Theoretische Grundlagen]{Theoretische Grundlagen} \label{chap:2_THEO}

\section{Grundlagen der akustischen Wahrnehmung}
\label{sec:2_GDAW}
    
    Hier wird ein kurzer Einleitungstext zu den Grundlagen der akustischen Wahrnehmung stehen.
    
    \subsection{Interaurale Zeitdifferenz}
    \label{subsec:2_ITD}
        
        Hier wird genauer auf die ITD eingegangen.
    
    \subsection{Interaurale Pegeldifferenz}
    \label{subsec:2_ILD}
    
        Und folglich hier nöher auf die ILD.

    
\section{Interpolationsverfahren}
\label{sec:2_IV}

Allgemeines zur Notwendigkeit der Interpolation.

    \subsection{Allgemeine Übersicht}
    \label{subsec:2_AUE}

    Hier erscheint eine Übersicht der "{}einfachen"{}  Interpolationsalgorithmen. Danach wird genauer auf die Interpolation im Frequenzbereich eingegangen.
    
    \subsection{Interpolation im Frequenzbereich}
    \label{subsec:ILD}

    
        \begin{enumerate}
            \item Blockschaltbild
            \item Mathematische Grundlagen
            \begin{enumerate}
                \item Fast Fourier Transform
                \item Short Time Fourier Transform
                \item Berechnung des Distanzkoeffizienten
            \end{enumerate}    
        \end{enumerate}

\section{Headtracking}
\label{sec:2_HT}

Hier wird die die Notwendigkeit des Headtrackings erläutert und zudem auf verschiedene Realisierungsmöglichkeiten eingegangen.