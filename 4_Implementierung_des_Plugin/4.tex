\chapter[Implementierung des Plugin]{Implementierung des Plugin} \label{chap:IMPL}


\section{Die Entwicklungsumgebung Unity}
\label{sec:4_DEU}

Hier werden allgemeine Informationen zu der Entwickungsumgebung Unity nieder geschrieben. Darunter fallen bspw. auch:

\begin{itemize}
    \item Platformunabhängigkeit
    \item Struktur der Entwicklungsumgebung (Szenen, Objekte, ...)
    \item Audio-Mixer
    \item Native Audio Plugin SDK
\end{itemize}


\section{Umsetzung unter Windows}
\label{sec:4_UUW}

Kombination einer Skriptsprache (hier C\#, im Unity Editor) sowie native Entwicklung (C++, Native Audio Plugin SDK, Verwendung von Visual Studio)

    \subsection{Erstellung eines Wiedergabesystems}
    \label{subsec:4_IDS}
    
    Nämlich so, dass immer die richtigen Schallquellen (abhängig von der Position des Ohres) auf die jeweiligen Ausgänge gelegt werden.
    
    \subsection{Digitale Signalverarbeitung}
    \label{subsec:4_DSV}
    
        \subsubsection{Filterung}
        \label{subsubsec:4_F}
        
        Hochpass und Tiefpass (Linkwitz-Riley 4. Ordnung)
        
        \subsubsection{Short Time Fourier Transform}
        \label{subsubsec:4_STFT}
        
        Transformation der Hochpassanteile, Überlagerung im Frequenzbereich.
    
        \subsection{Syntheseschritt}
        \label{subsec:4_SYN}

        Rücktransformation in den Zeitbereich, Überlagerung mit dem tiefrequenten Zeitsignal.

\section{Probleme unter Android}
\label{sec:4_PUA}

Latenz, Latenz, Latenz. Die Unity Audio Engine ist eben kein Mercedes Benz.